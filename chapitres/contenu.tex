%\setcounter{unbalance}{2}
\begin{multicols}{3}[\centerline{ \large\em \textbf{Formulaire personnel - Secure Embedded System}}][2cm]
%%%%%%%%%%%%%%%%%%%%%%%%%%%%%%%%%%%%%%%%%%%%%%%%%%%
{\Large \textbf{Buildroot}}
\paragraph*{1. D’expliquer les principaux répertoires de buildroot\\}
	
\begin{minipage}{\linewidth}
	\centering
    \includegraphics[width =0.8\columnwidth]{images/1.png}
    \includegraphics[width =0.8\columnwidth]{images/2.png}
\end{minipage}\\
Ce qui est manquant dans le dossier output sera recompil´e lorsque la commande make est lancée (ou alors en
faisant la commande make <package>-rebuild

%%%%%%%%%%%%%%%%%%%%%%
\paragraph*{2. D’expliquer le principe de fonctionnement de buildroot\\}
Générateur de Linux embarqué avec le système de cross-compilation\\
\begin{minipage}{\linewidth}
	\centering
    \includegraphics[width =0.8\columnwidth]{images/3.png}
\end{minipage}\\
1) The /buildroot/.config file contains the NanoPi Neo Plus 2 buildroot configuration\\
2) During the make, the configured packages sources files are downloaded to the
directories output/uboot, …\\
3) At the end of the make, the sdcard.img file is created. This file containts the bootloader, kernel, rootfs, …\\
4) The sdcard is flashed with the sdcard.img file\\
5) The sdcard is put in the NanoPi Neo Plus 2\\
%%%%%%%%%%%%%%%%%%%%%%
\paragraph*{3. D’expliquer la configuration de buildroot pour un hardware donné\\}
The Buildroot configuration is contained in two files: .config and xxx\_defconfig\\
- Buildroot uses Kconfig like the Linux kernel\\
- The .config file is a full default config file with more 4000 lines\\
- The defconfig stores only the values for options for which the non-default value is chosen. It is a small file \\
%%%%%%%%%%%%%%%%%%%%%%
\paragraph*{4. D’expliquer comment faire un patch et appliquer ce patch dans buildroot\\}

%%%%%%%%%%%%%%%%%%%%%%
\paragraph*{5. D’expliquer comment configurer, compiler buildroot, u-boot, kernel\\}
make menuconfig to config buildroot\\
make linux-menuconfig to config linux kernel\\
make uboot-menuconfig to config uboot\\
La commande make permet de compiler u-boot et buildroot\\
make linux-rebuild compile le linux\\
%%%%%%%%%%%%%%%%%%%%%%
\paragraph*{6. D’expliquer comment la SD-Card est générée\\}
\begin{minipage}{\linewidth}
	\centering
    \includegraphics[width =0.8\columnwidth]{images/5.png}
\end{minipage}\\
Pour créer sdcard.img, buildroot utilise le script genimage.sh \\
fichier : /buildroot/board/friendlyarm/scripts/genimages.cfg file\\
\begin{minipage}{\linewidth}
	\centering
    \includegraphics[width =0.8\columnwidth]{images/11.png}
\end{minipage}\\
%%%%%%%%%%%%%%%%%%%%%%
\paragraph*{7. D’expliquer comment le rootfs est généré\\}
- A rootfs skeleton is in the directory /buildroot/system/skeleton\\
- This skeleton is copied to the pseudo rootfs directory /buildroot/output/target\\
- It is possible to add files, directories with rootfs\_overlay\\
- After the make command, the pseudo rootfs is populated and copied to one file in this directory: \\
/buildroot/output/image/rootfs.xxx (xxx can be ext4,
squashfs, …)\\
\begin{minipage}{\linewidth}
	\centering
    \includegraphics[width =0.8\columnwidth]{images/6.png}
\end{minipage}\\
%%%%%%%%%%%%%%%%%%%%%%
\paragraph*{8. D’expliquer le rootfs\_overlay\\}
Le dossier rootfs\_overlay permet d’ajouter des fichiers
au rootfs
%%%%%%%%%%%%%%%%%%%%%%
\paragraph*{9. Savoir installer un nouveau package dans buildroot\\}


%%%%%%%%%%%%%%%%%%%%%%%%%%%%%%%%%%%%%%%%%%%%%%%%%%%
{\Large \textbf{U-boot}}\\
\paragraph*{10.D'expliquer le démarrage du NanoPi\\}
\begin{minipage}{\linewidth}
	\centering
    \includegraphics[width =0.8\columnwidth]{images/4.png}
\end{minipage}\\
Le démarrage du NanoPi NEO Plus2 se décompose en 6 phases:
- Lorsque le µP est mis sous tension, le code stocké dans son BROM va charger
dans ses 32KiB de SRAM interne le firmware « sunxi-spl » stocké dans le
secteur no 16 de la carte SD / eMMC et l’exécuter.\\
- Le firmware « sunxi-spl » (Secondary Program Loader) initialise les couches
basses du µP, puis charge l’U-Boot dans la RAM du µP avant de le lancer.\\
- L’U-Boot va effectuer les initialisations hardware nécessaires (horloges,
contrôleurs, …) avant de charger l’image non compressées du noyau Linux dans
la RAM, le fichier «Image», ainsi que le fichier de configuration FDT (flattened device tree).\\
- L’U-Boot lancera le noyau Linux en lui passant les arguments de boot (bootargs).\\
- Le noyau Linux procédera à son initialisation sur la base des bootargs et des éléments de configuration contenus dans le fichier FDT (sun50i-h5-nanopi-neoplus2.dtb).\\
- Le noyau Linux attachera les systèmes de fichiers (rootfs, tmpfs, usrfs, …) et poursuivra son exécution.
%%%%%%%%%%%%%%%%%%%%%%
\paragraph*{11. De connaître, expliquer les principales commandes de u-boot utilisées durant le démarrage\\}
Si on appuie sur une touche pendant le démarrage, on entre dans le mode u-boot\\
\textit{boot} load the Linux kernel, Image file, the FDT (Flattened Device Tree) and start Linux
\textit{booti} permet de lancer d'image linux.\\
\textit{mmc} mmc(MultiMediaCard) sub system\\
\textit{printenv} print environement variables\\
\begin{minipage}{\linewidth}
	\centering
    \includegraphics[width =0.8\columnwidth]{images/9.png}
\end{minipage}\\
%%%%%%%%%%%%%%%%%%%%%%
\paragraph*{12. De savoir comment configurer u-boot\\}
On configure avec make uboot-menuconfig puis on effectue la compilation avec une des deux mani`eres :
1. make uboot-rebuild
2. supprimer les fichiers puis make
La configuration de u-boot est stock´ee dans .config
%%%%%%%%%%%%%%%%%%%%%%
\paragraph*{13. D’expliquer comment améliorer la sécurité de u-boot\\}
%%%%%%%%%%%%%%%%%%%%%%
\paragraph*{14. De connaître les différentes étapes pour la création de l’image de u-boot.itb\\}
%%%%%%%%%%%%%%%%%%%%%%
\paragraph*{15. Savoir ce que fait la commande strip sur un fichier elf\\}
%%%%%%%%%%%%%%%%%%%%%%
\paragraph*{16. De connaître les différentes étapes pour la création de uImage\\}
%%%%%%%%%%%%%%%%%%%%%%
\paragraph*{17. De connaître l’utilité du Flattened Device Tree\\}
The Flattened Device-Tree (FDT) was introduced in kernel 2.6. It is a file which
contains the hardware description. Linux uses it for its configuration\\
 FDT has two files:\\
§ .dts: Device Tree Source, it is an ascii file\\
§ .dtb: Device Tree Blob, it is a binary file\\
After the introduction of FTD with the kernel 2.6, a new binary file format was
created: FIT (Flattened Image Tree).  This format allows to insert different files into a single file
%%%%%%%%%%%%%%%%%%%%%%
\paragraph*{18. De connaître de manière générale le mapping de la SDCard\\}

%%%%%%%%%%%%%%%%%%%%%%
\paragraph*{19. D’expliquer le fichier boot.scr\\}
Le fichier boot.scr est utilisé par u-boot pour charger le
kernel Linux. Il est créé avec la commande:\\
mkimage -C none -A arm64 -T script -d board/friendlyarm/nanopi-neoplus2/boot.cmd output/images/boot.scr
\begin{minipage}{\linewidth}
	\centering
    \includegraphics[width =0.8\columnwidth]{images/12.png}
\end{minipage}\\

%%%%%%%%%%%%%%%%%%%%%%%%%%%%%%%%%%%%%%%%%%%%%%%%%%%
{\Large \textbf{Compilation du noyau}}\\
\paragraph*{20. De connaître les principaux répertoires du noyau Linux\\}
This directory has these main sub-directories:\\
\textit{arch} Hardware dependent code\\
\textit{block} Generic functions for the block devices\\
\textit{crypto} Cryptographic algo. used in the kernel\\
\textit{Documentation} Documentation about the kernel\\
\textit{drivers} All drivers known by the kernel\\
\textit{fs} All filesystem know by the kernel\\
\textit{include} kernel include files\\\\
\textit{init} Init code (function start\_kernel)
\textit{ipc} Interprocess communication\\
\textit{kernel} Kernel code, scheduler, mutex, …\\
\textit{lib} different libraries used by the kernel\\
\textit{mm} Memory management\\
\textit{net} Different protocols, IPv4, IPv6, bluetooth, ...\\
\textit{samples} Different examples, kobject, kfifo, ...\\
\textit{security} Encrypted keys, SELinux, ...\\
\textit{sound} Sound support for Linux kernel\\
\textit{virt} Kernel-based virtual machine\\
This directory has these main files\\
\textit{vmlinux} Linux kernel, ELF format, ARM aarch64\\
\textit{.config} Linux kernel configuration\\
\textit{.config.old} Old Linux kernel configuration\\
\textit{Kconfig} Configuration for the make linux-xconfig\\
\textit{Makefile} makefile\\
%%%%%%%%%%%%%%%%%%%%%%
\paragraph*{21. De connaître les principales méthodes pour sécuriser le noyau Linux\\}
Enable -fstack-protector buffer overflow detection \\
%%%%%%%%%%%%%%%%%%%%%%
\paragraph*{22. D’expliquer le principe des software attacks : buffer overflow, ret2libc, ROP\\}
A buffer overflow attack can insert and executes a shell code in an executable
stack.\\
\begin{minipage}{\linewidth}
	\centering
    \includegraphics[width =0.8\columnwidth]{images/8.png}
\end{minipage}\\
Now, the stack memory is no longer executable\\
§ However, a technique called ret2libc could be used to bypass non executable stack memory.\\
The main idea is to execute code in an executable memory like libc() or other libraries.\\
§ Another technique called ROP, or Return-Oriented Programming allows also to bypass non executable stack memory. The main idea is to execute code in the program itself\\
\begin{minipage}{\linewidth}
	\centering
    \includegraphics[width =0.8\columnwidth]{images/7.png}
\end{minipage}\\
%%%%%%%%%%%%%%%%%%%%%%
\paragraph*{23. D’expliquer le principe des protections contre les softwares attacks : ASLR, PIE, canary\\}
ASLR (Address Space Layout Randomization) randomize\_va\_space randomizes
the stack and heap addresses\\
The PIE (Position Independent Executable) avoids the ret2lib and ROP problems
(because code addresses change)\\

\begin{minipage}{\linewidth}
	\centering
    %\includegraphics[width =0.8\columnwidth]{images/7.png}
\end{minipage}\\

%%%%%%%%%%%%%%%%%%%%%%%%%%%%%%%%%%%%%%%%%%%%%%%%%%%
{\Large \textbf{Valgrind}}\\
\paragraph*{24. De connaître les différents outils de Valgrind et leur utilisation\\}
1. Memcheck is a memory error detector. It helps you make your programs, particularly those written in C and C++, more correct.\\
2. Cachegrind is a cache and branch-prediction profiler. It helps you make your programs run faster.\\
3. Callgrind is a call-graph generating cache profiler. It has some overlap with Cachegrind, but also gathers some information that Cachegrind does not.\\
4. Helgrind is a thread error detector. It helps you make your multi-threaded programs more correct.\\
5. Massif is a heap and stack profiler. It helps you make your programs use less memory.\\
\begin{minipage}{\linewidth}
	\centering
    \includegraphics[width =0.8\columnwidth]{images/10.png}
\end{minipage}\\
%%%%%%%%%%%%%%%%%%%%%%
\paragraph*{25. Pour un code donné avec des erreurs, savoir quel-s outil-s de Valgrind utiliser\\}
%%%%%%%%%%%%%%%%%%%%%%
\begin{minipage}{\linewidth}
	\centering
    %\includegraphics[width =0.8\columnwidth]{images/num14.png}
\end{minipage}\\

%%%%%%%%%%%%%%%%%%%%%%%%%%%%%%%%%%%%%%%%%%%%%%%%%%%
{\Large \textbf{Hardening Linux}}\\
\paragraph*{26. De contrôler l’intégrité d’un package, d’un programme\\}
%%%%%%%%%%%%%%%%%%%%%%
\paragraph*{27. De configurer un nouveau package, programme\\}
%%%%%%%%%%%%%%%%%%%%%%
\paragraph*{28. De cross-compiler un programme\\}
%%%%%%%%%%%%%%%%%%%%%%
\paragraph*{29. De contrôler les services, les ports ouverts\\}
%%%%%%%%%%%%%%%%%%%%%%
\paragraph*{30. De contrôler les « file systems »\\}
%%%%%%%%%%%%%%%%%%%%%%
\paragraph*{31. De contrôler les permissions des fichiers, répertoires\\}
By separating file systems into various partitions, it is possible to fine tune permissions and functionalities.\\
Ajournaling file system must be installed (e.g. ext3, ext4) and activated\\
Areas where users have “write privileges” should be kept on their own partition.\\
LVM (Logical Volume Manager) can be used when more than four partitions are required.\\
§ read access (r)\\
§ write access (w)\\
§ execute access (x)\\
%%%%%%%%%%%%%%%%%%%%%%
\paragraph*{32. De sécuriser le réseau\\}
%%%%%%%%%%%%%%%%%%%%%%
\paragraph*{33. De contrôler-sécuriser les comptes utilisateurs\\}
%%%%%%%%%%%%%%%%%%%%%%
\paragraph*{34. De limiter le login root\\}
%%%%%%%%%%%%%%%%%%%%%%
\paragraph*{35. De sécuriser le noyau\\}
%%%%%%%%%%%%%%%%%%%%%%
\paragraph*{36. De sécuriser une application\\}
%%%%%%%%%%%%%%%%%%%%%%
\paragraph*{37. De contrôler le démarrage de Linux\\}
%%%%%%%%%%%%%%%%%%%%%%
\paragraph*{38. D’appliquer la méthodologie OSSTMM simplifiée\\}
%%%%%%%%%%%%%%%%%%%%%%
\begin{minipage}{\linewidth}
	\centering
    %\includegraphics[width =0.8\columnwidth]{images/num14.png}
\end{minipage}\\

%%%%%%%%%%%%%%%%%%%%%%%%%%%%%%%%%%%%%%%%%%%%%%%%%%%
{\Large \textbf{Filesystem}}\\
\paragraph*{39. De connaître les différents types de systèmes de fichiers ainsi que leurs applications\\}
\paragraph*{40. De connaître les caractéristiques des filesystems ext2-3-4, ainsi que les commandes associées\\}
\paragraph*{41. D’expliquer les différents « files systems » utilisés dans les systèmes embarqués (ext2-3-4,BTRFS, F2FS, NILFS2, XFS, ZFS, ...)\\}
\paragraph*{42. Expliquer les files system de type Journal, B\_Tree/CoW, log filesystem\\}
\paragraph*{43. De connaître les caractéristiques du filesystem Squashfs, ainsi que les commandes associées\\}
\paragraph*{44. De connaître les caractéristiques du filesystem tmpfs, ainsi que les commandes associées\\}
\paragraph*{45. De connaître les caractéristiques du filesystem LUKS, ainsi que les commandes associées\\}
\paragraph*{46. Savoir expliquer la gestion des clés de LUKS\\}
\paragraph*{47. De connaître les caractéristiques du filesystem InitramFS, ainsi que les commandes associées\\}
\paragraph*{48. De savoir créer un initramFS\\}
\begin{minipage}{\linewidth}
	\centering
    %\includegraphics[width =0.8\columnwidth]{images/num14.png}
\end{minipage}\\

%%%%%%%%%%%%%%%%%%%%%%%%%%%%%%%%%%%%%%%%%%%%%%%%%%%
{\Large \textbf{Filesystem security}}\\
\paragraph*{49. De connaître les « files permissions » sous Linux\\}
\paragraph*{50. De contrôler et sécuriser les comptes utilisateurs sous Linux\\}
\paragraph*{51. De connaître les real-effective userID and groupID\\}
\paragraph*{52. De connaître les ACL\\}
\paragraph*{53. De connaître les attributs particuliers des filesystems ext2-3-4\\}
\paragraph*{54. De rechercher des permissions de fichier faibles\\}
\paragraph*{55. Comment sécuriser les répertoires temporaires\\}
\paragraph*{56. De savoir comment les mots de passe sont mémorisés sous Linux\\}
\paragraph*{57. De connaitre les différentes possibilités pour casser un mot de passe\\}
\paragraph*{58. De savoir utiliser hashcat pour casser un mot de passe\\}
\begin{minipage}{\linewidth}
	\centering
    %\includegraphics[width =0.8\columnwidth]{images/num14.png}
\end{minipage}\\

%%%%%%%%%%%%%%%%%%%%%%%%%%%%%%%%%%%%%%%%%%%%%%%%%%%
{\Large \textbf{Firewall iptables}}\\
\paragraph*{59. De connaître les principes de Netfilter, iptables\\}
\paragraph*{60. Savoir expliquer les notions de chain-tables\\}
\paragraph*{61. Savoir expliquer les différences entre les firewall Stateless et Stateful\\}
\paragraph*{62. Savoir configurer avec iptables un firewall simple de types Stateless (pages 17-19) et Stateful (pages 26-32)\\}
\paragraph*{63. Connaître le principe des NFQUEUE\\}
\begin{minipage}{\linewidth}
	\centering
    %\includegraphics[width =0.8\columnwidth]{images/num14.png}
\end{minipage}\\

%%%%%%%%%%%%%%%%%%%%%%%%%%%%%%%%%%%%%%%%%%%%%%%%%%%
{\Large \textbf{TPM}}\\
\paragraph*{64. Savoir expliquer uniquement le principe des chiffrements symétrique, asymétrique,fonctions de hachage, la signature digitale\\}
\paragraph*{65. Connaitre les différentes implémentations des TPM (discrete, integrated, Hypervisor,Software)\\}
\paragraph*{66. Connaitre l’architecture interne d’un TPM\\}
\paragraph*{67. Connaitre les différentes hiérarchies des TPM (endorsement, platform, owner, null)\\}
\paragraph*{68. Savoir créer, utiliser des clés avec un TPM\\}
\paragraph*{69. Connaitre les commandes principales d’un TPM (pas tous les paramètres, mais savoir expliquer ce que font ces commandes, être capable de dessiner ce que font les
commandes)\\}
\paragraph*{70. Savoir encrypter-décrypter, signer-vérifier avec un TPM\\}
\paragraph*{71. Savoir utiliser les registres PCR\\}
\paragraph*{72. Savoir sauver des données sur le TPM\\}
\paragraph*{73. Savoir sauver des données et les protéger avec une PCR policy\\}
\begin{minipage}{\linewidth}
	\centering
    %\includegraphics[width =0.8\columnwidth]{images/num14.png}
\end{minipage}\\

\end{multicols}