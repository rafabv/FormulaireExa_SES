%\setcounter{unbalance}{2}
\begin{multicols}{3}[\centerline{ \large\em \textbf{Formulaire personnel - Secure Embedded System}}][2cm]
%%%%%%%%%%%%%%%%%%%%%%%%%%%%%%%%%%%%%%%%%%%%%%%%%%%
{\Large \textbf{Buildroot}}\\
\textbf{1. D’expliquer les principaux répertoires de buildroot\\}
	
\begin{minipage}{\linewidth}
	\centering
    \includegraphics[width =0.6\columnwidth]{images/1.png}
    \includegraphics[width =0.6\columnwidth]{images/2.png}
\end{minipage}\\
Ce qui est manquant dans le dossier output sera recompilé lorsque la commande make est lancée (ou alors en faisant la commande make <package>-rebuild\\
%%%%%%%%%%%%%%%%%%%%%%
\textbf{2. D’expliquer le principe de fonctionnement de buildroot\\}
Générateur de Linux embarqué avec le système de cross-compilation\\
\begin{minipage}{\linewidth}
	\centering
    \includegraphics[width =0.8\columnwidth]{images/3.png}
\end{minipage}\\
1) The /buildroot/.config file contains the NanoPi Neo Plus 2 buildroot configuration\\
2) During the make, the configured packages sources files are downloaded to the
directories output/uboot, …\\
3) At the end of the make, the sdcard.img file is created. This file containts the bootloader, kernel, rootfs, …\\
4) The sdcard is flashed with the sdcard.img file\\
5) The sdcard is put in the NanoPi Neo Plus 2 \\
%%%%%%%%%%%%%%%%%%%%%%
\textbf{3. D’expliquer la configuration de buildroot pour un hardware donné\\}
The Buildroot configuration is contained in two files: .config and xxx\_defconfig\\
- Buildroot uses Kconfig like the Linux kernel\\
- The .config file is a full default config file with more 4000 lines\\
- The defconfig stores only the values for options for which the non-default value is chosen. It is a small file 
%%%%%%%%%%%%%%%%%%%%%%
\\ \textbf{4. D’expliquer comment faire un patch et appliquer ce patch dans buildroot\\}
Initialiser un repo git dans le folder où on va faire les modifs\\
Commit\\
Faire les modifs puis commit\\
Demander la diff entre les commit\\
git format-patch main –o my\_patches\\
Faire un dossier my\_patches dans board .. nano et quand on fait git format-patch .. il va enregistrer le patch dans notre dossier
%%%%%%%%%%%%%%%%%%%%%%
\\ \textbf{5. D’expliquer comment configurer, compiler buildroot, u-boot, kernel\\}
make menuconfig to config buildroot\\
make linux-menuconfig to config linux kernel\\
make uboot-menuconfig to config uboot\\
La commande make permet de compiler u-boot et buildroot\\
make linux-rebuild compile le linux
%%%%%%%%%%%%%%%%%%%%%%
\\
\textbf{6. D’expliquer comment la SD-Card est générée\\}
\begin{minipage}{\linewidth}
	\centering
    \includegraphics[width =0.8\columnwidth]{images/5.png}
\end{minipage}\\
Pour créer sdcard.img, buildroot utilise le script genimage.sh \\
fichier : /buildroot/board/friendlyarm/scripts/genimages.cfg file\\
\begin{minipage}{\linewidth}
	\centering
    \includegraphics[width =0.8\columnwidth]{images/11.png}
\end{minipage}
%%%%%%%%%%%%%%%%%%%%%%
\\ \textbf{7. D’expliquer comment le rootfs est généré\\}
- A rootfs skeleton is in the directory /buildroot/system/skeleton\\
- This skeleton is copied to the pseudo rootfs directory /buildroot/output/target\\
- It is possible to add files, directories with rootfs\_overlay\\
- After the make command, the pseudo rootfs is populated and copied to one file in this directory: \\
/buildroot/output/image/rootfs.xxx (xxx can be ext4,
squashfs, …)\\
\begin{minipage}{\linewidth}
	\centering
    \includegraphics[width =0.8\columnwidth]{images/6.png}
\end{minipage}
%%%%%%%%%%%%%%%%%%%%%%
\\ \textbf{8. D’expliquer le rootfs\_overlay\\}
Le dossier rootfs\_overlay permet d’ajouter des fichiers
au rootfs (executables cross-compilé, fichiers de configuration)
%%%%%%%%%%%%%%%%%%%%%%
\\ \textbf{9. Savoir installer un nouveau package dans buildroot\\}
(info dans LinuxHardening)\\
Known buildroot packages are in the directory /buildroot/packages\\
- Example with a generic foo package\\
- The directory /buildroot/packages/foo contains mainly these files:
\begin{itemize}
\item Config.in file, written in kconfig language, describing the configuration options for the package
\item foo.mk file describing where to fetch the source, how to build and install it, etc. \\
Optional foo.hash file, providing hashes to check the integrity of the
downloaded tarballs and license files
\item Sxx\_foo file, it is the start script for the foo package
\end{itemize}
%%%%%%%%%%%%%%%%%%%%%%%%%%%%%%%%%%%%%%%%%%%%%%%%%%%
\newpage
{\Large \textbf{U-boot}}
\\ \textbf{10.D'expliquer le démarrage du NanoPi\\}
Le démarrage du NanoPi NEO Plus2 se décompose en 6 phases:\\
- Lorsque le µP est mis sous tension, le code stocké dans son BROM va charger
dans ses 32KiB de SRAM interne le firmware « sunxi-spl » stocké dans le
secteur no 16 de la carte SD / eMMC et l’exécuter.\\
- Le firmware « sunxi-spl » (Secondary Program Loader) initialise les couches
basses du µP, puis charge l’U-Boot dans la RAM du µP avant de le lancer.\\
- L’U-Boot va effectuer les initialisations hardware nécessaires (horloges,
contrôleurs, …) avant de charger l’image non compressées du noyau Linux dans
la RAM, le fichier «Image», ainsi que le fichier de configuration FDT (flattened device tree).\\
- L’U-Boot lancera le noyau Linux en lui passant les arguments de boot (bootargs).\\
- Le noyau Linux procédera à son initialisation sur la base des bootargs et des éléments de configuration contenus dans le fichier FDT (sun50i-h5-nanopi-neoplus2.dtb).\\
- Le noyau Linux attachera les systèmes de fichiers (rootfs, tmpfs, usrfs, …) et poursuivra son exécution.
\begin{minipage}{\linewidth}
	\centering
    \includegraphics[width =0.8\columnwidth]{images/4.png}
    \includegraphics[width =0.6\columnwidth]{images/38.png}
\end{minipage}\\
%%%%%%%%%%%%%%%%%%%%%%
\\ \textbf{11. De connaître, expliquer les principales commandes de u-boot utilisées durant le démarrage\\}
Si on appuie sur une touche pendant le démarrage, on entre dans le mode u-boot\\
\textit{boot} load the Linux kernel, Image file, the FDT (Flattened Device Tree) and start Linux
\textit{booti} permet de lancer d'image linux.\\
\textit{mmc} mmc(MultiMediaCard) sub system\\
\textit{printenv} print environement variables\\
\begin{minipage}{\linewidth}
	\centering
    \includegraphics[width =0.6\columnwidth]{images/9.png}
\end{minipage}
%%%%%%%%%%%%%%%%%%%%%%
\\ \textbf{12. De savoir comment configurer u-boot\\}
On configure avec make uboot-menuconfig puis on effectue la compilation avec une des deux manières :\\
1. make uboot-rebuild\\
2. supprimer les fichiers puis make\\
La configuration de u-boot est stockée dans .config
%%%%%%%%%%%%%%%%%%%%%%
\\ \textbf{13. D’expliquer comment améliorer la sécurité de u-boot\\}
Il est possible d'ajouter deux options :\\
- suppress the –g option: delete the debug information\\
- -fstack-protector-all : adds extra code to check buffer overflows, such as stack smashing attacks. This is done by adding a guard variable to functions. This variable is called Canary
%%%%%%%%%%%%%%%%%%%%%%
\\ \textbf{14. De connaître les différentes étapes pour la création de l’image de u-boot.itb\\}
mkimage reads u-boot.its and builds u-boot.itb\\
- u-boot-nodtb.bin: u-boot code. Created during command make. Raw file with only the loadable sections of the u-boot file\\
- bl31.bin: trust zone.\\
- sun501-h5 … .dtb: Flattened device tree (device Tree Blob)\\
\begin{minipage}{\linewidth}
	\centering
    \includegraphics[width =0.8\columnwidth]{images/13.png}
\end{minipage}\\
\begin{minipage}{\linewidth}
	\centering
    \includegraphics[width =0.8\columnwidth]{images/16.png}
\end{minipage}\\
Le fichier u-boot.its va chercher les 3 fichiers mentionnés ci-dessus:\\
\begin{minipage}{\linewidth}
	\centering
    \includegraphics[width =0.8\columnwidth]{images/14.png}
    \includegraphics[width =0.8\columnwidth]{images/15.png}
\end{minipage}
%%%%%%%%%%%%%%%%%%%%%%
\\ \textbf{15. Savoir ce que fait la commande strip sur un fichier elf\\}
Un fichier elf c'est un fichier executable linux dans lequel il y a des informations comme une table de symbole et des informations de débugging. Quand on strip un elf, on enlève ses infos ce qui rend le fichier plus petit mais surtout beaucoup plus dur a reverse engineer (++ sécurité)
%%%%%%%%%%%%%%%%%%%%%%
\\ \textbf{16. De connaître les différentes étapes pour la création de uImage\\}
Image linux au format uboot.\\
Préparation des fichiers nécessaires: Il est nécessaire de disposer des fichiers suivants: Le noyau Linux compilé (vmlinux ou zImage), le système de fichiers racine (rootfs), le script de démarrage (bootscript)\\
Utilisation de l'outil mkimage: Cet outil est fourni avec le noyau Linux et permet de créer l'image uImage à partir des fichiers précédemment préparés. \\
La commande générale pour créer une image uImage est :\\
$mkimage -A <architecture> -O <os> -T <type> -C <compression> -a <load_address> -e <entry\_ point> -n <image\_name> -d <image\_file> <uImage\_file>$\\
Une fois l'image uImage créée, elle doit être copiée sur la mémoire flash du système embarqué, généralement en utilisant l'outil de programmation de mémoire flash spécifique à la plate-forme cible. 
%%%%%%%%%%%%%%%%%%%%%%
\\ \textbf{17. De connaître l’utilité du Flattened Device Tree\\}
It is a file which contains the hardware description. Linux uses it for its configuration\\
FDT has two files:\\
- .dts: Device Tree Source, it is an ascii file\\
- .dtb: Device Tree Blob, it is a binary file\\
After the introduction of FTD with the kernel 2.6, a new binary file format was created: FIT (Flattened Image Tree).  This format allows to insert different files into a single file
%%%%%%%%%%%%%%%%%%%%%%
\\ \textbf{18. De connaître de manière générale le mapping de la SDCard\\}
cat /buildroot/board/friendlyarm/scripts/genimages.cfg\\

\begin{minipage}{\linewidth}
	\centering
    \includegraphics[width =1\columnwidth]{images/52.png}
\end{minipage}\\
\begin{minipage}{\linewidth}
	\centering
    \includegraphics[width =0.2\columnwidth, angle = 90]{images/58.png}
\end{minipage}
%%%%%%%%%%%%%%%%%%%%%%
\\ \textbf{19. D’expliquer le fichier boot.scr\\}
Le fichier boot.scr est utilisé par u-boot pour charger le
kernel Linux. Il est créé avec la commande:\\
mkimage -C none -A arm64 -T script -d board/friendlyarm/nanopi-neoplus2/boot.cmd output/images/boot.scr\\
\begin{minipage}{\linewidth}
	\centering
    \includegraphics[width =0.8\columnwidth]{images/12.png}
\end{minipage}
\columnbreak

%%%%%%%%%%%%%%%%%%%%%%%%%%%%%%%%%%%%%%%%%%%%%%%%%%%
{\Large \textbf{Compilation du noyau}}\\ 
\textbf{20. De connaître les principaux répertoires du noyau Linux\\}
This directory has these main sub-directories:\\
\textit{arch} Hardware dependent code\\
\textit{block} Generic functions for the block devices\\
\textit{crypto} Cryptographic algo. used in the kernel\\
\textit{Documentation} Documentation about the kernel\\
\textit{drivers} All drivers known by the kernel\\
\textit{fs} All filesystem know by the kernel\\
\textit{include} kernel include files\\\\
\textit{init} Init code (function start\_kernel)\\
\textit{ipc} Interprocess communication\\
\textit{kernel} Kernel code, scheduler, mutex, …\\
\textit{lib} different libraries used by the kernel\\
\textit{mm} Memory management\\
\textit{net} Different protocols, IPv4, IPv6, bluetooth, ...\\
\textit{samples} Different examples, kobject, kfifo, ...\\
\textit{security} Encrypted keys, SELinux, ...\\
\textit{sound} Sound support for Linux kernel\\
\textit{virt} Kernel-based virtual machine\\
This directory has these main files :\\
\textit{vmlinux} Linux kernel, ELF format, ARM aarch64\\
\textit{.config} Linux kernel configuration\\
\textit{.config.old} Old Linux kernel configuration\\
\textit{Kconfig} Configuration for the make linux-xconfig\\
\textit{Makefile} makefile
%%%%%%%%%%%%%%%%%%%%%%
\\ \textbf{21. De connaître les principales méthodes pour sécuriser le noyau Linux\\}
- -Overflow detection : activer -fstack-protector. Protection contre erreurs de overflow de mémoire. \\
- Platform selection : La désactivation des platforms non utilisés permettent de restreindre d'accès au linux embarqué\\
- Kernel .config support : Désactiver cette option va faire que le fichier .config n'ira pas dans l'image envoyée sur la cible. Les configurations du linux embarquée ne seront donc pas disponibles sur la cible\\
- Randomize espace :  activer, rendre le système imprédictible\\
- Randomize SLAB Allocator : gestion de la mémoire de façon plus efficace et aussi de le rendre moins prédictible\\
- Randomize address of kernel Image : adressage de la mémoire moins prédictible\\
- Write protect kernel text and module : empêche le code d'être modifié. Le code du kernel sera en mode read-only\\
- Enable random number generator : Permet au système de générer des nombres aléatoires ce qui est utile lors des processus d'encryptage (génération des clés)\\
- Filter access : restreit l'accès à la mémoire qu'au user root\\
- Restrict unprivileged access to the kernel syslog : restreit l'accès des logs du système qu'au user root\\
- Kernel Memory Initialization : initialise toutes les valeur de la mémoire à zéro évitant ainsi les problème de heap memory non initialisée\\
- File system : active les labels de securité pour les différentes évolution de format de partition\\
%%%%%%%%%%%%%%%%%%%%%%
\\ \textbf{22. D’expliquer le principe des software attacks : buffer overflow, ret2libc, ROP\\}
A buffer overflow attack can insert and executes a shell code in an executable
stack.\\
\begin{minipage}{\linewidth}
	\centering
    \includegraphics[width =0.4\columnwidth]{images/8.png}
\end{minipage}\\
Now, the stack memory is no longer executable\\
- ret2libc could be used to bypass non executable stack memory. The main idea is to execute code in an executable memory like libc() or other libraries.\\
- ROP, or Return-Oriented Programming allows also to bypass non executable stack memory. The main idea is to execute code in the program itself\\
\begin{minipage}{\linewidth}
	\centering
    \includegraphics[width =0.8\columnwidth]{images/7.png}
\end{minipage}
%%%%%%%%%%%%%%%%%%%%%%
\\ \textbf{23. D’expliquer le principe des protections contre les softwares attacks : ASLR, PIE, canary\\}
ASLR (Address Space Layout Randomization) randomize\_va\_space randomizes
the stack and heap addresses\\
The PIE (Position Independent Executable) : Functions and variables addresses of an executable are not static and are computed during the program execution. This option avoids the ROP problem.\\

%%%%%%%%%%%%%%%%%%%%%%%%%%%%%%%%%%%%%%%%%%%%%%%%%%%
\newpage
{\Large \textbf{Valgrind}}
\\ \textbf{24. De connaître les différents outils de Valgrind et leur utilisation\\}
1. Memcheck is a memory error detector. It helps you make your programs, particularly those written in C and C++, more correct.\\
2. Cachegrind is a cache and branch-prediction profiler. It helps you make your programs run faster. Comment j'utilise la cache -> optimizer lecture d'un gros fichier.\\
3. Callgrind is a call-graph generating cache profiler. It has some overlap with Cachegrind, but also gathers some information that Cachegrind does not. Comment j'utilise le cpu.\\
4. Helgrind is a thread error detector. It helps you make your multi-threaded programs more correct. Un soft est bloqué par exemple, on peut utiliser helgrind pour essayer de trouver ou ça bloque ou si ça risque de bloquer quelque part.\\
5. Massif is a heap and stack profiler. It helps you make your programs use less memory. Permet de vérifier d'utilisation de la heap -> optimiser la heap, éviter l'overflow\\
\begin{minipage}{\linewidth}
	\centering
    \includegraphics[width =0.6\columnwidth]{images/10.png}
\end{minipage}
%%%%%%%%%%%%%%%%%%%%%%
\\ \textbf{25. Pour un code donné avec des erreurs, savoir quel-s outil-s de Valgrind utiliser\\}
voir point 24\\
\columnbreak

%%%%%%%%%%%%%%%%%%%%%%%%%%%%%%%%%%%%%%%%%%%%%%%%%%%
{\Large \textbf{Hardening Linux}}
\\ \textbf{26. De contrôler l’intégrité d’un package, d’un programme\\}
En premier télécharger le programme foo et sa signature ensuite il faut vérifier l'intégrité avec:\\
gpg --verify foo.tar.gz.asc\\
Il nous répond qu'il peut pas en donnant une RSA key: par ex. 23948UT5. Donc il faut importer cette clé publique:\\
gpg --keyserver keyserver.ubuntu.com --search-keys 23948UT5\\
Et relancer la vérification et c'est bon.
%%%%%%%%%%%%%%%%%%%%%%
\\ \textbf{27. De configurer un nouveau package, programme\\}
\begin{minipage}{\linewidth}
	\centering
    \includegraphics[width =0.8\columnwidth]{images/17.png}
\end{minipage}
S'il n'existe pas encore dans buildroot (donc une version inférieure n'a jamais été installé) on le place dans /buildroot/packages.
Et on va faire les choix de versions dans le foo.mk et ensuite on le configure avec make menuconfig le choix de config sont décris dans Config.in
%%%%%%%%%%%%%%%%%%%%%%
\\ \textbf{28. De cross-compiler un programme\\}
Utiliser le compilateur généré par buildroot (/buildroot/output/host/bin/)\\
Le compilateur et tous les outils de cross compil sont regroupés dans le programme toolchain-wrapper (un peu comme busybox pour uboot; /buildroot/output/host/bin/toolchain-wrapper) généré par buildroot une fois qu'on a configuré l'architecture sur laquelle notre image devra tourner.\\
%%%%%%%%%%%%%%%%%%%%%%
\textbf{29. De contrôler les services, les ports ouverts\\}
\textit{netstat} command shows the open tcp and udp ports\\
$netstat -atun$\\
\textit{lsof} command shows open files\\
$ lsof | grep IP$\\
\textit{nmapssss} command scans open ports for another hosts\\
$nmap -Pn -p 1-65535 192.168.0.11$ (scan all tcp ports fot the host 192.168.0.11)
%%%%%%%%%%%%%%%%%%%%%%
\\ \textbf{30. De contrôler les « file systems »\\}
By separating file systems into various partitions, it is possible to fine tune permissions and functionalities
%%%%%%%%%%%%%%%%%%%%%%
\\ \textbf{31. De contrôler les permissions des fichiers, répertoires\\}
By separating file systems into various partitions, it is possible to fine tune permissions and functionalities.\\
Ajournaling file system must be installed (e.g. ext3, ext4) and activated\\
Areas where users have “write privileges” should be kept on their own partition.\\
LVM (Logical Volume Manager) can be used when more than four partitions are required.\\
- read access (r)\\
- write access (w)\\
- execute access (x)
%%%%%%%%%%%%%%%%%%%%%%
\\ \textbf{32. De sécuriser le réseau\\}
- Disable IPv6 :\\
sysctl –w /net/ipv6/conf/default/disable\_ipv6=1\\
- IP source routing must be disabled : \\
sysctl –w net/ipv4/conf/all/accept\_source\_route=0\\
- Forwarding (Routing) of normal and multicast packets should also be deactivated unless expressively needed : \\
sysctl –w net/ipv4/conf/all/forwarding=0\\
sysctl –w net/ipv6/conf/all/forwarding=0\\
sysctl –w net/ipv4/conf/all/mc\_forwarding=0\\
sysctl –w net/ipv6/conf/all/mc\_forwarding=0\\
- Block ICMP redirect messages : \\
sysctl –w net/ipv4/conf/all/accept\_redirects=0\\
sysctl –w net/ipv6/conf/all/accept\_redirects=0\\
sysctl –w net/ipv4/conf/all/secure\_redirects=0\\
sysctl –w net/ipv4/conf/all/send\_redirects =0\\
- Enable source route verification in order to prevent spoofing (attaque, usurpation d'identité) : \\
sysctl –w net/ipv4/conf/default/rp\_filter=0\\
- Log all malformed packed and ignore icmp bogus ones:\\
sysctl –w net/ipv4/conf/all/log\_martians=1\\
sysctl –w net/ipv4/icmp\_ignore\_bogus\_error\_responses=1\\
- Disable ICMP echo and timestamp responses sent via broadcast or multicast:\\
sysctl –w net/ipv4/icmp\_echo\_ignore\_broadcasts=1\\
- Increase resilience under heavy TCP load by increasing backlog buffer and by enabling syn cookies:\\
sysctl –w net/ipv4/tcp\_max\_syn\_backlog=4096\\
sysctl –w net/ipv4/tcp\_syncookies=1\\
- \textit{sysctl –p} command reads the file /etc/sysctl.conf and configure kernel parameters\\
\begin{minipage}{\linewidth}
	\centering
    \includegraphics[width =0.6\columnwidth]{images/18.png}
\end{minipage}
%%%%%%%%%%%%%%%%%%%%%%
\\ \textbf{33. De contrôler-sécuriser les comptes utilisateurs\\}
The umask is the user file creation right\\
\#umask 027 (all permissions)\\
- Le premier chiffre contrôle les droits d'accès de l'utilisateur propriétaire.\\
- Le deuxième chiffre contrôle les droits d'accès d'un groupe d'utilisateurs.\\
- Le troisième chiffre contrôle les droits d'accès des autres utilisateurs.\\
0 : rwx; 1 : rw-; 2 : r-x; 3 : r--; 4 : -wx; 5 : -w-; 6 : --x; 7 : - - -(aucun)
%%%%%%%%%%%%%%%%%%%%%%
\\ \textbf{34. De limiter le login root}
\begin{itemize}
\item Direct root logins must be restricted to the console (emergency situations usually require
hands at the console). This can be achieved by modifying the /etc/securetty file.
Example for NanoPi: echo “ttyS0” > /etc/securetty
\item Only root can access to the root directory: chmod 700 /root
\item Use su or sudo commands in order to have the root rights
\item The root PATH must not contain the current directory or writable directories
\end{itemize}~
%%%%%%%%%%%%%%%%%%%%%%
\textbf{35. De sécuriser le noyau}~
\begin{itemize}
\item ASLR (Adresse Space Layout Randomization) randomize\_va\_space: 1: shared libraries will be loaded to random addresses, echo 2 = echo1 plus heap randomization\\
\begin{minipage}{\linewidth}
	\centering
    \includegraphics[width =0.8\columnwidth]{images/19.png}
\end{minipage}
\item Write protect kernel text section, kernel configuration:\\
\begin{minipage}{\linewidth}
	\centering
    \includegraphics[width =0.8\columnwidth]{images/20.png}
\end{minipage}
\item Strip assembler-generated symbols during link, kernel configuration\\
\begin{minipage}{\linewidth}
	\centering
    \includegraphics[width =0.8\columnwidth]{images/21.png}
\end{minipage}
\item Enable -fstack-protector buffer overflow detection , kernel configuration\\
\begin{minipage}{\linewidth}
	\centering
    \includegraphics[width =0.8\columnwidth]{images/22.png}
\end{minipage}
\item Only root can access to the kernel system logs (through dmesg).\\
\begin{minipage}{\linewidth}
	\centering
    \includegraphics[width =0.8\columnwidth]{images/23.png}
\end{minipage}
\end{itemize}~
%%%%%%%%%%%%%%%%%%%%%%
\textbf{36. De sécuriser une application\\}
Dans le makefile :\\
CFLAGS="-fPIE -fstack-protector-all -D\ \_FORTIFY\ \_SOURCE=2 -ftrapv"\\
LDFLAGS="-Wl,-z,now,-z,relro -z,noexecstack, -pie "\\
Ou en ligne de commande : \\
gcc –Wall -Wextra -z noexecstack -pie -fPIE -fstack-protector-all -
Wl,-z,relro,-z,now –O –D\_FORTIFY\_SOURCE=2 –ftrapv -o test test.c
\begin{itemize}
\item noexecstack : no-execute stack. Do not allow execution of instructions stored on the stack. An operating system and/or hardware which supports the NX bit may mark certain areas of memory as non-executable (stack, heap)
\item pie : voir question 23
\item stack-protector-all : Stack smashing protection is a way to protect programs from stack buffer overflows by adding random values (canaries) between the function’s local variables and the saved instruction pointer
\item relro, now : All dynamic symbols must be resolved during the program startup. The GOT (Global Offset Table) table is initialized and next it is marked read-only. This prevents GOT overwrite attacks.
\item FORTIFY\_SOURCE : this macros provide buffer overflow checks for the following functions \\
memcpy, mempcpy, memmove, memset, strcpy, stpcpy, strncpy, strcat, strncat, sprintf, vsprintf, snprintf, vsnprintf, gets\\
Set to 1, with compiler optimization level 1 (gcc -O1) and above,
checks that shouldn’t change the behavior of conforming programs are performed\\
Set to 2 some more checking is added, but some conforming programs might fail\\
Some of the checks can be performed at compile time, the results are in compiler warnings; other checks take place at run time, the results are in a run-time error if the check fails. 
\item ftrapv : Integer overflow. not short or unsigned int.
\item strnXXX functions : Replace strcat, strcpy by strn.... functions. \\
char *strncpy(char *dest, const char *src, size\_t n)\\
This function is similar to strcpy(), except that at most n bytes of src are copied. If the length of src is less than n, strncpy() pads the remainder of dest with null bytes. \\
Some C functions such as gets(), strcpy(), strcat(), printf() are known to be
insecure because they don’t check the size of the destination buffers
\end{itemize}~
%%%%%%%%%%%%%%%%%%%%%%
\textbf{37. De contrôler le démarrage de Linux\\}
Utiliser le TPM?!
%%%%%%%%%%%%%%%%%%%%%%
\\ \textbf{38. D’appliquer la méthodologie OSSTMM simplifiée\\}
Open Source Security Testing Methodology Manual (OSSTMM) provides a
methodology for a thorough security test\\
A set of security metrics, called Risk Assessment Values (RAVs), provides a
powerful tool that can provide a graphical representation of state, and shows
changes in state over the time\\
\begin{minipage}{\linewidth}
	\centering
    \includegraphics[width =0.4\columnwidth]{images/24.png}
\end{minipage}\\
\begin{itemize}
\item Access: Count each access which potentially allow interaction with the embedded system. Count each TCP/UDP ports really open\\
nmap -Pn –n -p 1-65535 IP\\
nmap –Pn –n –sU -p 1-65535 IP\\
netstat -atunp
\item Authentication: Count each connection to a system (ssh, console, …) which asks a username and password
\item Confidentiality: Count each connection where the data is encrypted. Example: it is better to have only a ssh connnection than a telnet connection to a system
\item Vulnerability, Weekness, Concern:\\
Check if a version of a program has vulnerabilities\\
Check the authentication passwords. The passwords are stored in /etc/shadow,
you can use hashcat, John the Ripper, hydra programs in order to check the
passwords
Check the robustness of the cryptographic algorithms
\item Exposure: Give direct or indirect unjustified visibility of targets\\
nmap –Pn –n –p 22 –sV IP // indicate the service version
(Openssh 8.1p)
\end{itemize}
%%%%%%%%%%%%%%%%%%%%%%%%%%%%%%%%%%%%%%%%%%%%%%%%%%%
\newpage
{\Large \textbf{Filesystem}}
\\ \textbf{39. De connaître les différents types de systèmes de fichiers ainsi que leurs applications\\}
Pour les systèmes embarqués, il existe deux catégories de systèmes de fichiers, les volatiles en RAM et les persitents sur des Flash (NOR et de plus en plus NAND)\\
Deux technologies principales sont disponible sur les Flash, soit les MTD (Memory Technology Device) ou les MMC/SD-Card (Multi-Media-Card / Secure Digital Card)\\

\begin{minipage}{\linewidth}
	\centering
    \includegraphics[width =\columnwidth]{images/25.png}
\end{minipage}
\\ \textbf{40. De connaître les caractéristiques des filesystems ext2-3-4, ainsi que les commandes associées}
 \begin{itemize}
 \item EXT2 :  file system for the Linux kernel. Not a journaled file system. Uses block mapping in order to reduce file fragmentation (it allocates several free blocks).\\
 After an unexpected power failure or system crash (also called an
unclean system shutdown), each mounted ext2 file system on the
machine must be checked for consistency with the e2fsck program
 \item EXT3 : replaces ext2. Is compatible with ext2. Is a journaled file system. The ext3 file system prevents loss of data integrity even when an unclean system shutdown occurs.
 \item EXT4 : Compatible with ext3 and ext2, making it possible to
mount ext3 and ext2 as ext4. Supports Large file system (volume max : $2^{60}$ bytes, file max : $2^{40}$ bytes). Ext4 uses extents (zone de stockage contiguë réservée pour un fichier sur le système de fichiers d'un ordinateur) (as opposed to the traditional block mapping scheme used by ext2 and ext3), which improves performance when using large files and reduces metadata overhead for large files\\
Commands :\\
\begin{minipage}{\linewidth}
	\centering
    \includegraphics[width =0.6\columnwidth]{images/26.png}
\end{minipage}
\end{itemize}
filesystem options can be activated with the mount command (or to the
/etc/fstab file)\\
Journaling: A journaling file system is a file system that keeps track of changes not yet committed to the file system's. The journaling guarantees the data consistency, but it reduces the file system performances\\
MMC/SD-Card constraints: In order to improve the longevity of MMC/SDCard, it is necessary to reduce the unnecessary writes\\
Mount options:
\begin{itemize}
\item noatime : Do not update inode (data structure in a Unix-style file system that describes a file-system object such as a file or a directory) access times on this filesystem (e.g., for faster access on the news spool to speed up news servers)
\item nodiratime : Do not update directory inode access times on this filesystem
\item relatime : this option can replace the noatime and nodiratime if an application needs the access time information (like mutt)
\end{itemize}
Mount options fot the journaling (man ext4):
\begin{itemize}
\item Data=journal : All data is committed into the journal prior to being written into the main filesystem (It is the safest option in terms of data integrity and reliability, though maybe not so much for performance)
\item Data=ordered :  This is the default mode. All data is forced directly out to the main file system before the metadata being committed to the journal
\item Data=writeback :  Data ordering is not preserved - data may be written into the main filesystem after its metadata has been committed to the journal
\end{itemize}
Autres commandes:\\
\begin{minipage}{\linewidth}
	\centering
    \includegraphics[width =0.8\columnwidth]{images/27.png}
\end{minipage}\\
File /etc/fstab contains descriptive information about the filesystems the
system can mount
\\ \textbf{41. D’expliquer les différents « files systems » utilisés dans les systèmes embarqués (ext2-3-4,BTRFS, F2FS, NILFS2, XFS, ZFS, ...)}~
\begin{itemize}
\item EXT2,3,4 : point 40
\item BTRFS : point 42
\item F2FS : Flash-Friendly File System. It is a log filesystem. It can be tuned using many parameters to allow best handling on different supports. F2FS features: Atomic operations, Defragmentation, TRIM support (reporting free blocks for reuse)
\item NILFS2 : New Implementation of a Log-structured File System. Uses log filesystem and B-Tree technologies. Userspace garbage collector.
\item XFS : Journaling filesystem. Supports huge filesystems. Designed for scalability (ability to increase or decrease). Does not seem to be handling power loss (standby state) well.
\item ZFR : Zettabyte ($10^{21}$)File System. B-Tree file system. Provides strong data integrity. Supports huge filesystems. Not intended for embedded systems (requires RAM).
\item Squashfs : compressed read-only filesystem for Linux.  uses gzip, lzma, lzo, lz4 and xz compression to compress files, inodes (data structure in a Unix-style file system that describes a file-system object such as a file or a directory) and directories. Intended for general read-only filesystem use, for archival use, and in embedded systems with small processors where low overhead is needed.
\end{itemize}
Performances : EXT4 is currently the best solution for embedded systems using MMC; F2FS and NILFS2 show very good write performances\\
Features : BTRFS is a next generation filesystem; NILFS2 provides simpler but similar features\\
Scalability : EXT4 clearly doesn’t scale as well as BTRFS and F2FS\\
\begin{minipage}{\linewidth}
	\centering
    \includegraphics[width =0.5\columnwidth]{images/28.png}
\end{minipage}
\\ \textbf{42. Expliquer les files system de type Journal, B\_Tree/CoW, log filesystem}
\begin{itemize}
\item File system : soit l'organisation hiérarchique des fichiers au sein d'un système d'exploitation. Soit l'organisation des fichiers au sein d'un volume physique ou logique, qui peut être de différents types (par exemple NTFS, FAT, FAT32, ext2fs, ext3fs, ext4fs, zfs, btrfs, etc.)
\item Journalized file system : keeps track of every modification in a journal in a dedicated area. Allows to restore a corrupted filesystem. Modifications are first recorded in the journal then applied on the disk. If a corruption occurs: The File System will either keep or drop the modifications. Journalized filesystems : EXT3, EXT4, XFS, Reiser4
\item B-TREE/CoW : B+ tree is a data structure that generalized binary trees. CoW (Copy on Write) is used to ensure no corruption occurs at runtime (the original storage is never modified. When a write request is made, data is written to a new storage area; original storage is preserved until modifications are committed; if an interruption occurs during writing the new storage area, original storage can be used). Filesystems using CoW : ZFS, BTRFS, NILFS2\\
\begin{minipage}{\linewidth}
	\centering
    \includegraphics[width =\columnwidth]{images/29.png}
\end{minipage}
\item Log filesystem : Log-structured filesystems use the storage medium as circular buffer and new blocks are always written to the end. Log-structured filesystems are often used for flash media since they will naturally perform wear-levelling (extend the life of solid-state). The log-structured approach is a specific form of copy-on-write behavior. Log filesystems : F2FS, NILFS2, JFFS2, UBIFS\\
\begin{minipage}{\linewidth}
	\centering
    \includegraphics[width =0.6\columnwidth]{images/30.png}
\end{minipage}\\
1) Initial state\\
2) Block 1-3 are updated, old blocks 1-3 are not used\\
3) Garbage copies block2 and 4, and frees old block1-2-3-4
\item BTRFS (B-Tree filesystem) : BTRFS is a “new” file system compared to EXT. It is a B-Tree filesystem.
\end{itemize} ~
\textbf{43. De connaître les caractéristiques du filesystem Squashfs, ainsi que les commandes associées\\}
Create the squashed file system dir.sqsh for the regular directory /data/config/:\\
bash\# mksquashfs /data/config/ /data/dir.sqsh\\
\begin{minipage}{\linewidth}
	\centering
    \includegraphics[width =0.5\columnwidth]{images/31.png}
\end{minipage}\\
The mount command is used with a loopback device in order to read the squashed
file system dir.sqsh\\
bash\# mkdir /mnt/dir\\
bash\# mount –o loop –t squashfs /data/dir.sqsh /mnt/dir\\
bash\# ls /mnt/dir\\
It is possible to copy the dir.sqsh to an unmounted partition (e.g. /dev/sdb2) with the dd command and next to mount the partition as squashfs file system\\
bash\# umount /dev/sdb2\\
bash\# dd if=dir.sqsh of=/dev/sdb2\\
bash\# mount /dev/sdb2 /mnt/dir -t squashfs\\
bash\# ls /mnt/dir\\
Squashfs - NanoPi - Buildroot : \\
\begin{minipage}{\linewidth}
	\centering
    \includegraphics[width =\columnwidth]{images/32.png}
\end{minipage}
\\ \textbf{44. De connaître les caractéristiques du filesystem tmpfs, ainsi que les commandes associées\\}
Tmpfs is a file system which keeps all files in virtual memory (temporarily transferring data from random access memory (RAM) to disk storage).\\
Everything in tmpfs is temporary in the sense that no files will be created on
your hard drive. If you unmount a tmpfs instance, everything stored therein
is lost\\
devtmpfs (device temporary) is a file system with automatically populates nodes files (/dev/…) known by the kernel. \\
 mount -n -t devtmpfs devtmpfs /dev\\
/dev is automatically populated by the kernel with its known devices\\
fstab: table des différents systèmes de fichiers sur un ordinateur sous Unix/Linux
\\ \textbf{45. De connaître les caractéristiques du filesystem LUKS, ainsi que les commandes associées\\}
Linux Unified Key Setup is the standard for Linux hard disk encryption. Data in the LUKS partition is encrypted. Data used in the user space is decrypted by dmcrypt.\\
\begin{minipage}{\linewidth}
	\centering
    \includegraphics[width =0.6\columnwidth]{images/33.png}
\end{minipage}\\
Luks features : compatibility via standardization, secure against attacks, support for multiple keys, effective passphrase revocation, free\\
cryptsetup is a utility used to configure dmcrypt\\
cryptsetup uses the /dev/random and /dev/urandom node file\\
dmcrypt (Device-mapper) crypts target and provides transparent encryption of block devices using the kernel crypto API (kernel configuration, Cryptographic API)\\
To enable dmcrypt :\\
\begin{minipage}{\linewidth}
	\centering
    \includegraphics[width =\columnwidth]{images/34.png}
\end{minipage}\\
To use cryptsetup, it is required to add a new package in buildroot:\\
\begin{minipage}{\linewidth}
	\centering
    \includegraphics[width =0.8\columnwidth]{images/35.png}
\end{minipage}\\
LUKS with NanoPi: On the SD Card, create a third partition (with fdisk or parted). A passphrase generates\\

plein de commands................\\
%%%%%%%%%%%%%%%%%%%%%%
\textbf{46. Savoir expliquer la gestion des clés de LUKS\\}~
\begin{minipage}{\linewidth}
	\centering
    \includegraphics[width =0.8\columnwidth]{images/36.png}
\end{minipage}\\
TKS1 uses Argon2 or PBKDF2 (Password-Based Key Derivation Function 2)
method in order to provide a better resistance against brute force attacks
based on entropy weak user passphrase
%%%%%%%%%%%%%%%%%%%%%%
\\ \textbf{47. De connaître les caractéristiques du filesystem InitramFS, ainsi que les commandes associées\\}
initramfs is a root filesystem that is loaded at an early stage of the boot
process\\
It provides early userspace commands which lets the system do things that
the kernel cannot easily do by itself during the boot process.\\
Boot sequence\\
Les points: 1) execute sunxi-spl,2) load and launch U-Boot et 3) load Image and fdt sont les mêmes avec ou sans initramf (voir point 10)\\
En 4) u-boot copie le initramfs en RAM\\
En 5) u-boot démarre le kernel Linux\\
En 6) le kernel Linux s’initialise\\
En 7) le kernel Linux mount le initramfs et exécute le script /init, qui peut
contenir différentes commandes, par exemple: décrypter le rootfs qui est
dans une partition LUKS\\
En 8) le script /init active le rootfs\\
\begin{minipage}{\linewidth}
	\centering
    \includegraphics[width =0.6\columnwidth]{images/37.png}
    \includegraphics[width =0.3\columnwidth]{images/39.png}
\end{minipage}\\
1) Kernel (Image), initramfs (uInitrd), flattened device tree (Sun50i…) and boot.scr files are located in the partition 1 of the SDCard\\
2) Kernel, initramfs, Sun50i.. are copied to the RAM\\
3) Kernel mounts initramfs (uInitrd file)\\
4) Kernel executes init script stored in initramfs. This init script can execute early different commands\\
5) Init script executes the command switch\_root, which switches to the standard rootfs located in the partition 2 and executes the /sbin/init
command
\begin{minipage}{\linewidth}
	\centering
    \includegraphics[width =\columnwidth]{images/40.png}
\end{minipage}
%%%%%%%%%%%%%%%%%%%%%%
\\ \textbf{48. De savoir créer un initramFS\\}
Principle to build an initramfs:\\
1. to copy the right files into a directory (/buildroot/ramfs),\\
2. to copy these files in a cpio archive file,\\
3. to compress this file\\
4. To add the uboot header\\
Initramfs: kernel configuration :\\
make linux-menuconfig\\
CONFIG\_BLK\_DEV\_INITRD=y\\
General setup ---> [*] Initial RAM filesystem and RAM disk (initramfs/initrd) support\\
Automount a devtmpfs and initiate the /dev:\\
Device Drivers $\rightarrow$ Generic Drivers options $\rightarrow$ Maintain a devtmpfs filesystem to mount at /dev $\rightarrow$ Automount a devtmpfs\\
%%%%%%%%%%%%%%%%%%%%%%%%%%%%%%%%%%%%%%%%%%%%%%%%%%%
{\Large \textbf{Filesystem security}}
\\ \textbf{49. De connaître les « files permissions » sous Linux\\}
voir point 33. Write (create-delete-rename) file/folder.
%%%%%%%%%%%%%%%%%%%%%%
\\ \textbf{50. De contrôler et sécuriser les comptes utilisateurs sous Linux\\}
?????????????\\ 
%%%%%%%%%%%%%%%%%%%%%%
\textbf{51. De connaître les real-effective userID and groupID\\}
User: Each user is assigned a unique number, called a user ID, or UID\\
Group: Each group is assigned a unique number, called a group ID, or
GID. Every group contains one or more user IDs. A single user ID can be
a member of some of groups, but groups can’t contain other groups; they
can contain only users. Like users, groups have names.\\
Sticky bit : A directory that has the sticky bit set allows you to delete a file only if you are the file owner
\\ \textbf{52. De connaître les ACL\\}
Access Control List. Filesystems that allow the ACL: Ext2, ext3, ext4, ReiserFS, JFS, XFS, Btrfs, Tmpfs, JFFS2, CIFS. 
\\ \textbf{53. De connaître les attributs particuliers des filesystems ext2-3-4\\}
-i :file/directory can not be modified, deleted, renamed or linked
symbolically, not even by root. Only root or a binary with the necessary
rights can set this attribute.\\
-A :Date of last access is not updated (only useful for reducing disk
access)\\
-S :File is synchronous, the records in the file are made immediately on
the disc. (equivalent to the sync option of mount)\\
-a :File can only be open in append mode for writing (log files, etc.) Only
redirection >> can be used, the file can not be deleted.\\
-c :File is automatically compressed before writing to disk, and unpacked
before playback\\
-d :File will not be saved by the dump\\
-j :Ext3-ext4 :A file with the 'j' attribute has all of its data written to
the ext3 or ext4 journal before being written to the file itself\\
-s : When the file is destroyed, all data blocks are being released to zero.
\\ \textbf{54. De rechercher des permissions de fichier faibles\\}
\#find . -perm 200 // file permissions = 200\\
\#find . -perm -220 // write bit for user and group = 1\\
\#find . -perm /220 // write bit for user or group = 1\\
\#find . -perm +220 // write bit for user or group = 1 (like /220)\\
“other” write bit = 1\\
\# find / -type d -perm -2 -ls (find the directories with the others can write)\\
\# chmod -R o-w /dir1/dir2 (Be careful: remove other bit = write for all files under /dir1/dir2)\\
SUID bit = 1\\
\# find / -type f -perm -4000 -ls\\
\# chmod u-s /usr/bin/file\\
\# chmod -R u-s /var/directory/ \\
GUID bit = 1\\
\# find / -type f -perm -2000 -ls\\
\# chmod g-s /usr/bin/file\\
\# chmod -R g-s /var/directory/ \\
Sticky bit = 1\\
\# find / -type f -perm -1000 -ls\\
\# chmod -t /usr/bin/file\\
\# chmod -R -t /var/directory/ 
\\ \textbf{55. Comment sécuriser les répertoires temporaires\\}
voir point 44\\
/tmp, /var/tmp are directories or partitions which hold temporary files\\
These directories can store temporary bots, malware, rootkits, ...\\
A more secure solution would be to set /tmp in its own partition, so that it can be mounted independently of the / partition and have more restrictive options set. Exemple :\\
/dev/sda7 /tmp ext4 nosuid,noexec,nodev,rw 0 0\\
This would set the nosuid, noexec, and nodev options, meaning: no suid programs are permitted, nothing can be executed, and no device files (node file) may exist (see man mount)\\
In order to improve the security for the tmpfs, the /etc/fstab file must by modified\\
tmpfs /dev/shm tmpfs mode=0777 0 0\\
tmpfs /tmp tmpfs $\textbackslash $defaults,nosuid,noexec,nodev,rw 0 0\\
If you don’t have the ability to create a fresh /tmp partition on existing
drives, you can use the loopback capabilities of the Linux kernel by creating a loopback filesystem that will be mounted as /tmp and can use the same
restrictive mount options
\\ \textbf{56. De savoir comment les mots de passe sont mémorisés sous Linux\\}
On a host, the encryption method (DES, MD5, SHA512) used is in the file
/etc/login.defs\\
 On NanoPi with busybox, the encryption method used is given by the command:\\
passwd –help\\
passwd -a des, or passwd -a md5 or passwd -a sha256 or passwd -a sha512
username
\\ \textbf{57. De connaitre les différentes possibilités pour casser un mot de passe}
\begin{itemize}
\item Dictionary attacks : uses a file with words called dictionnary. 
\item Combinator attack :  uses two dictionaries and tries all combinations with these 2 dictionaries: dict1= {1234, abcd}, dict2={toto, ruru} $\rightarrow$ 1234toto, 1234rutu, abcdtoto,abcdruru
\item Hybrid attacks :  uses a dictionary and add some letters/numbers at the end of each word: asdf12, asdf123, asdf1234, …
\item Brute force attacks : Attempt to try all possible combinations, which means it will take the longest amount of time out of the three types.
\item Mask attacks : Similar to Brute Force, this requires additional user inputs. If you know the length of the password, a few of the characters used, or even a prefix or suffix, it takes less time to recover.
\end{itemize}~
\textbf{58. De savoir utiliser hashcat pour casser un mot de passe\\}~
Hashcat attack modes :
\begin{itemize}
\item Straight mode : dictionary attack with one password database\\
hashcat -a 0 -m 0 hashfile dictionaryFile (MD5)\\
Hashcat reads each line of dictionaryFile, computes the MD5 hash and compares with hashes stored in hashfile
\item Combinator mode : hybrid attack with two password databases\\
hashcat -a 1 -m 0 hashfile dictionaryFile1 dictionaryFile2 (MD5)\\
combines dictionaryFile1 words with dictionaryFile1 words\\
\begin{minipage}{\linewidth}
	\centering
    \includegraphics[width =0.4\columnwidth]{images/41.png}
\end{minipage}
\item Brute/mask mode : advanced brute force-mask attack\\
 hashcat -a 3 -m 0 hashfile ?d?d?d\\
Mask attack : Try all combinations from a given keyspace just like in Brute-force attack, but more specific.
\item Hybrid mode :  hybrid attack (password database + mask)\\
hashcat -a 6 -m 0 hashfile ?u?u dictionaryFile ?d?d?d\\
One side is simply a dictionary, the other is the result of a Brute-Force/mask attack
\end{itemize} 
Hash mode : hash structure (0 : MD5, 1800: sha512 unix)\\
\vfill

\columnbreak

%%%%%%%%%%%%%%%%%%%%%%%%%%%%%%%%%%%%%%%%%%%%%%%%%%%
{\Large \textbf{Firewall iptables}}\\
\textbf{59. De connaître les principes de Netfilter, iptables\\}
iptables is a user-space application program that allows to configure the
tables provided by the Linux kernel firewall \\
%%%%%%%%%%%%%%%%%%%%%%
\\ \textbf{60. Savoir expliquer les notions de chain-tables\\}
When a packet arrives, it traverses different chains. Theses chains are contained in different tables. The combination chain-table are the hooks : Chains (INPUT, OUTPUT, FORWARD, PREROUTING, POSTROUTING) and tables (raw, mangle, nat, filter)\\
Chain explanation: INPUT(the packet is for the local host), OUTPUT (the packet is sent from the local host), FORWARD (the packet arrives to an interface (eth0) and it is forwarded to another interface (eth1)), PREROUTING/POSTROUTING (used by NAT)
Tables:
\begin{itemize}
\item Filter : mainly used for the firewalls. Contains the built-in chains : INPUT (for packets destined to local sockets), FORWARD (for packets being routed through the box), OUTPUT (for locally-generated packets).
\item mangle : This table is used for specialized packet alteration-modification. It consists of five built-ins chains: INPUT (for packets coming into the box itself), FORWARD (for altering packets being routed through
the box), POSTROUTING (for altering packets as they are about
to go out), PREROUTING (for altering incoming packets before
routing), OUTPUT (for altering locally-generated packets before
routing)
\item nat : Network Address Translation. This table is consulted when a packet that creates a new connection is encountered. It consists of three built-ins chains: PREROUTING (for altering packets as soon as they come in), OUTPUT (for altering locally-generated packets before routing), POSTROUTING (for altering packets as they are about to go out)
\end{itemize}
\begin{minipage}{\linewidth}
	\centering
    \includegraphics[width =0.4\columnwidth, angle = 90]{images/42.png}
\end{minipage}
%%%%%%%%%%%%%%%%%%%%%%
\\ \textbf{61. Savoir expliquer les différences entre les firewall Stateless et Stateful packet filtering\\}
- Stateless: Stateless packet filtering (table filter et ACCEPT,
DROP, REJECT). Permet de protéger au niveau
réseau (bloquage d’une ip en particulier, ou ports).
Les paquets sont analysés de maniére individuelle \\
- Statefull:  Permet de protéger au
niveau du paquet en fonction du contexte (précédents
paquets). Il est possible d’accepter des paquets venant
de l’extérieur seulement si ils sont des réponses à des
requêtes venant de l’intérieur\\
• Utilisation de connection tables pour traiter les
différentes parties des protocoles.\\
• NEW : Nouveau paquet qui n’est pas lié à une
connexion active\\
• ESTABLISHED : Une connexion passe de NEW
à ESTABLISHED losrque la connexion est validée
par la direction opposée\\
• RELATED : Paquets qui ne font pas partie d’une
connexion existante mais qui sont liés à une autre.
(Par exemple réponses ICMP pour une communication FTP).\\
\textbf{62. Savoir configurer avec iptables un firewall simple de types Stateless (pages 17-19) et Stateful (pages 26-32)\\}
NanoPi allows only incoming SSH connections\\
\begin{minipage}{\linewidth}
	\centering
    \includegraphics[width =0.8\columnwidth]{images/43.png}
\end{minipage}\\
NanoPi allows only incoming SSH connections and all outgoing connections\\
\begin{minipage}{\linewidth}
	\centering
    \includegraphics[width =0.8\columnwidth]{images/44.png}
\end{minipage}
\\ \textbf{63. Connaître le principe des NFQUEUE\\}
The NFQUEUE target means to pass the packet to userspace.\\
\columnbreak

%%%%%%%%%%%%%%%%%%%%%%%%%%%%%%%%%%%%%%%%%%%%%%%%%%%
{\Large \textbf{TPM (trusted platform module)}}
TPM generates strong, secure cryptographic keys\\
\begin{minipage}{\linewidth}
	\centering
    \includegraphics[width =0.6\columnwidth]{images/53.png}
\end{minipage}\\
\\ \textbf{64. Savoir expliquer uniquement le principe des chiffrements symétrique, asymétrique,fonctions de hachage, la signature digitale\\}

Cryptography -> needs a key to decryptograph
- Chiffrement symétrique : Une seul clé pour crypter et décrypter. \\
Block mode : Cleartext is divided in block. Each block has the same length (64, 128 bits). Each block is encrypted by an algorithm (AES, IDEA, 3DES, …)\\
\begin{minipage}{\linewidth}
	\centering
    \includegraphics[width =0.6\columnwidth]{images/45.png}
\end{minipage}\\
CBC mode : Cipher Block Chaining. Every block is coded with the result of the previous block.\\
\begin{minipage}{\linewidth}
	\centering
    \includegraphics[width =0.6\columnwidth]{images/46.png}
\end{minipage}\\
- Chiffrement asymétrique : A key for the encryption (public or private key). Another key for the decryption (private or public key). The public keys must be exchanged (stored in a certificate).\\
Confidentiality :Encrypt with the public key and decrypt with the private key (confidentiality, integrity). Digital signature : Encrypt with the private key and decrypt with the public key (Authenticity, integrity)
\\ \textbf{65. Connaitre les différentes implémentations des TPM (discrete, integrated, Hypervisor,Software)\\}
1. Discrete TPMs are dedicated chips that implement TPM functionalities in their own tamper resistant semiconductor package\\
2. Integrated TPMs are part of another chip. Intel: Platform Trust Technology (PTT), AMD: fTPM, ARM: TrustZone\\
3. Hypervisor TPMs (vTPMs) are virtual TPMs provided by and rely on hypervisors.\\
4. Software TPMs are software emulators of TPMs that run with no more protection than a regular program gets within an operating system
\\ \textbf{66. Connaitre l’architecture interne d’un TPM\\}
\begin{minipage}{\linewidth}
	\centering
    \includegraphics[width =0.8\columnwidth]{images/47.png}
\end{minipage}\\
RSA 1024/2048, ECC: Asymmetric algorithms, encrypt-decrypt, sign\\
AES: Symmetric algorithm, encrypt-decrypt, sign\\
SHA-256, SHA-1: hash function\\
Random generator: create random value\\
Key generator: Create key for asymmetric algo\\
NV Ram (Persistent area): Store different objects (keys, data, …) in NV Ram\\
PCR (Platform Configuration Registers) stores hash values of different parts: code, files, partitions, …\\
RAM (Transient area): Store keys, data, this area is limited (free this area with : tpm2\_flushcontext –t)
%%%%%%%%%%%%%%%
\\ \textbf{67. Connaitre les différentes hiérarchies des TPM (endorsement, platform, owner, null)\\}
- Endorsement hierarchy: The endorsement hierarchy is reserved for objects created and certified by the TPM manufacturer. The endorsement seed (eseed) is randomly generated at manufacturing
time and never changes during the lifetime of the device. The primary endorsement key is certified by the TPM manufacturer, and because its seed never changes, it can be used to identify the device\\
- Platform hierarchy: The platform hierarchy is reserved for objects created and certified by the OEM (Original Equipment Manufacturer) that builds the host platform. The platform seed (pseed) is randomly generated at manufacturing time, but can be changed by the OEM by calling tpm2\_changepps\\
- Owner hierarchy, also known as storage hierarchy: The owner hierarchy is reserved for us - the primary users of the TPM. When a user takes a TPM, the owner hierarchy can be erased by the tpm2\_clear command. tpm2\_clear generates a new owner seed (oseed)\\
- Null hierarchy: The null hierarchy is reserved for ephemeral keys. The null seed is re-generated each time the host reboots
%%%%%%%%%%%%%%%
\\ \textbf{68. Savoir créer, utiliser des clés avec un TPM\\}
\begin{minipage}{\linewidth}
	\centering
    \includegraphics[width =0.8\columnwidth]{images/48.png}
\end{minipage}\\
After tpm2\_createprimary command, public-private keys are stored in RAM in the transient area\\
- tpm2\_getcap handles-transient command shows the index of the public-private keys\\
- tpm2\_flushcontext command deletes keys from RAM (tpm2\_flushcontext 0x80000000 ou tpm2\_flushcontext -t)\\
- tpm2\_readpublic command get the public key from the TPM (tpm2\_readpublic -c 0x80000000(ou o\_primary.ctx) --format PEM -o o\_primary.public)\\
openssl command shows the public key (openssl rsa -pubin -in o\_primary.public -text)\\
%%%%%%%%%%%%%%%
\textbf{69. Connaitre les commandes principales d’un TPM (pas tous les paramètres, mais savoir expliquer ce que font ces commandes, être capable de dessiner ce que font les commandes)\\}
voir point 68\\
- child\_private file contains the private asymmetric key or symmetric keys. These keys are encrypted by the parent keys. 
- child\_public file contains public asymmetric keys and different references\\
Commands :\\
tpm2\_create -C o\_primary -G rsa2048 -u child\_public –r child\_private\\
- Child asymmetric keys:\\
\begin{minipage}{\linewidth}
	\centering
    \includegraphics[width =0.6\columnwidth]{images/49.png}
\end{minipage}\\
First load the key:\\
tpm2\_load -C primary.ctx -u child\_pub -r child\_pr -c child\\
Then ncrypt with child key: \\
tpm2\_rsaencrypt -c child -s rsaes clearfile –o encryptedfile\\
Or decrypt with Child keys :\\
tpm2\_rsadecrypt -c child -s rsaes encryptedfile -o clearfile\\
Or Verify signature with Child keys :\\
tpm2\_verifysignature -c child -g sha256 -s file.sign -m file\\
Create Child symmetric key :\\
\begin{minipage}{\linewidth}
	\centering
    \includegraphics[width =0.6\columnwidth]{images/57.png}
\end{minipage}\\
\textbf{70. Savoir encrypter-décrypter, signer-vérifier avec un TPM\\}
voir point 69\\ 
Decrypt (asymmetric key) :\\
\begin{minipage}{\linewidth}
	\centering
    \includegraphics[width =0.6\columnwidth]{images/54.png}
\end{minipage}\\
Sign (asymmetric key) :\\
\begin{minipage}{\linewidth}
	\centering
    \includegraphics[width =0.6\columnwidth]{images/55.png}
\end{minipage}\\
Verify (asymmetric key) :\\
\begin{minipage}{\linewidth}
	\centering
    \includegraphics[width =0.6\columnwidth]{images/54.png}
\end{minipage}\\
\textbf{71. Savoir utiliser les registres PCR\\}
Platform Configuration Registers. The prime use case is to provide a method to cryptographically record (measure) software state or configuration data used by a device. \\ 
\textbf{72. Savoir sauver des données sur le TPM\\}
\begin{minipage}{\linewidth}
	\centering
    \includegraphics[width =0.6\columnwidth]{images/50.png}
\end{minipage}
\\ \textbf{73. Savoir sauver des données et les protéger avec une PCR policy\\}
\begin{minipage}{\linewidth}
	\centering
    \includegraphics[width =0.6\columnwidth]{images/51.png}
\end{minipage}

\end{multicols}